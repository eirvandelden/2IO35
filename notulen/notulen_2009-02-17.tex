\documentclass[11pt,oneside,a4paper]{article}
\usepackage[dutch]{babel}

\begin{document}
\textbf{{\LARGE{Agenda Vergadering 3 februari 2009}}}\\ \\
\textbf{Voorzitter}: Dion\\
\textbf{Notulist}: Edin\\
\begin{enumerate}
    \item \textbf{Opening} 14:13
        \subitem \textbf{Vaststellen aan- en afwezigheid}
            \subsubitem Jeroen afwezig
    \item \textbf{Definitief vaststellen van de agenda}
        \subsubitem -
    \item \textbf{Notulen}
        \subitem \textbf{Bespreken van de notulen van de vorige vergadering}
            \subsubitem Notulen mailen
%    \item Ingekomen stukken
%    \item Geplande mededelingen
    \item \textbf{Overige mededelingen}
        \subitem \textbf{Tutor}
            \subsubitem - Feedback opdracht 1 gekregen:
              \begin{itemize}
                \item A' is een modelconstante, maar staat niet als dusdanig gedefinieerd.
                \item Bij de specificatie van groep B is de post conditie true
                \item implementatie van groep A is dus te sterk, skip is voldoende
                \item implementatie van groep B is fout, want hij moet nil returnen ipv -1
              \end{itemize}
            \subsubitem - Introductie voor de modules mag uitgebreid
            \subsubitem - Als er gebruik gemaakt moet worden van sockets, moet dat duidelijk gemaakt worden
            \subsubitem - Het protocol moet duidelijk worden gespecificeerd
            \subsubitem - Het is raadsaam ook scenario's te maken voor dingen die niet mogen gebeuren.
              
        \subitem \textbf{Groep}
            \subsubitem -
    \item \textbf{Planning}
        \subitem \textbf{Opdracht 2}
        \subsubitem Gebruik van sockets toegestaan mits duidelijk gespecificeerd
    \item \textbf{W.v.t.t.k}
        \subitem -
    \item \textbf{Sluiting} 14:37
\end{enumerate}
\end{document} 