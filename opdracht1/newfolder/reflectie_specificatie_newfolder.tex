\documentclass[a4paper,twoside,11pt]{article}
\usepackage{a4wide,amsmath,amssymb,lists,verbatim}
%\usepackage{sidecap}

\begin{document}
   \begin{titlepage}
        {\ }\\[5.0cm]
        { {\Large Verslag OGO 2.2 voorjaar 2009}}\\[0.2cm]
        {\bf \Huge Reflectie specificatie \& implementatie}\\[0.1cm]
        { {\Large {Technische informatica, TU/e} }}\\[1.0cm]
        {\ Eindhoven, \today }\\[0.2cm]
        \begin{flushright}
            {\bf {\small Groep 2b }}\\[0.0cm]
            {\em {\small Edin Dudojevic, 0608206}}\\
            {\em {\small Dion Jansen, 0590077 }}\\
            {\em {\small Anson van Rooij, 0596312}}\\[0.5cm]
        \end{flushright}
    \end{titlepage}
\section{Specification}
\subsection{General}
\begin{itemize}
    \item The use of for example $A$ and $A'$ before- and after-states of variable $A$ is not part of Hoare-tripling. It's part of the specification language $\mathbb{Z}$ and is meaningless in the way the group used it in their specification.
    \item As stated before, the group used $A$ and $A'$ to denote before- and after-states, but they used it wrong. Instead of assigning a value to $A'$ to denote an after-state they assigned a value to $A$ in the post-conditions.
\end{itemize}
\subsection{Add}
\begin{itemize}
    \item $(\neg n \inlist A)$ was used in the post-condition, where $n \in \mathbb{Z}$. What is a non-integer?
\end{itemize}
\subsection{Sort}
\begin{itemize}
    \item As stated before $A$ and $A'$ were used wrong. The post-condition of this procedure required the ordering of $A$, but since $A'$ is the after-state $A'$ is supposed to be ordered instead. Now we could order the array $A$ and return $A'$ which is, according to the precondition, the same as the unordered $A$ in the pre-condition.
\end{itemize}
\subsection{Median}
\begin{itemize}
    \item No $A'$ used in post-condition, so no after-state?
    \item Function will return the wrong element from the array A. When the number of elements in $A$ is even the function will return the element just above the middle element.
\end{itemize}

\section{Implementation}
\subsection{Code}
\begin{verbatim}
function middelste_derde_macht(A: Array of Integer): Integer;
begin
    middelste_derde_macht := -1
end;
\end{verbatim}
\subsection{Comments}
No comments. Due to our small fault in the specification, this implementation is correct $\ddot\smile$.
\end{document} 